
\chap Princip fungování BLDC motoru

%- fizikální princip fungování - lorentzova síla
%- řídící obvod
%- základní myšlenka řízení BLDC motoru

Tato kapitola se zaměřuje na popis fyzikální principu fungování BLDC motoru, popisu
 řídícího obvod motoru a základní myšlenky pro řízení BLDC motoru.

%https://www.st.com/content/dam/applications-pages/field-oriented-control.jpg

\sec Fyzikální princip fungování BLDC motoru

Vytváření točivého momentu v BLDC motoru je založeno na interakci magnetického pole generovaného statorovými cívkami s magnetickým polem rotoru.

%- Ampérův zákon
%- selenoid - magnetické pole
%- selenoid s feromagnetyckým jádrem
%- interakce magnetického pole s permanentními magnety vytváří točivý moment

\secc Magnetické pole statoru

Generování magnetického pole statoru je možné na základě Ampérova zákona, dle kterého proud procházejícím vodičem vyvolávává magnetické pole, 
jehož směr je kolmý na směr proudu. 
Hodnotu magnetického toku\fnote{Magnetický tok je vektorová veličina. Vyjadřuje množství magnetického pole procházející danou plochou. }
 tohoto pole lze vyjádřit pomocí Hopkinsona zákona jako:

$$ \muppphi = G_{m} U_{m}  \eqno (4.1) $$

Kde $ G_{m} $ je magnetická vodivost a $U_{m}$ je magnetomotorické napětí. Magnetická vodivost vyjadřuje schopnost 
materiálu vést magnetický tok a platí pro ni vztah $ G_{m} = \frac{\mupmu S}{l} $, kde $ \mupmu $ je permeabilita prostředí, 
S průřez cívky a l je její délka.
V případě cívky platí, že magnetomotorické napětí je rovno součinu proudu procházejícího cívkou s počtem závitů dané cívky. Tedy $ U_{m} = N I $.
Při vynásobení rovnice počtem závitů cívky se vyjádří vztah pro její celkový spřažený tok se všemi závity.

$$ \mumppsi = N \mupphi = G_{m} N^2 I  \eqno (4.2) $$

Výraz $N^2 G_{M}$ vyjadřuje vlastní indukčnost cívky. Tato veličina se značí jako L a vyjadřuje schopnost cívky vytvářet magnetické pole.
Celkový magnetický tok cívky tak lze vyjádřit jako:

$$ \muppsi = L I  \eqno (4.3) $$

Z tohoto vztahu je zřejmé, že magnetický tok ovlivňuje nejen velikost a směr proudu, ale i vlastní indukčnost cívky a prostředí ve kterém se cívka nachází.
V případě, že je v prostředí feromagnetické jádro, je magnetické pole vytvářené cívkou výrazně zesíleno a toho se využívá u 
drážkových BLDC motorů.

\medskip
\picw=8 cm \cinspic Pictures/4princip/coil-magnetic-flux.png 
\caption/f Znázornění magnetického pole cívky
\medskip

% Elektrotechnika I - Antonín Blahovec vydání paté strana 170-173

\secc Interakce magnetického pole

Statorové cívky tak vytvářejí magnetické pole, které následně interaguje s polem permanentních magnetů. 
V případě rozdílných směrů magnetické indukce\fnote{Magnetická indukce je veličina vyjadřující magnetické pole v 
daném bodě prostoru a je definována jako $\mbfB~=~\mbfPhi \setminus S $.}
 těchto polí dochází k točivého momentu, který se tyto pole snaží srovnat.

\medskip
\picw=7 cm \cinspic Pictures/4princip/magneticky-tok.png 
\caption/f Znázornění magnetického indukce statoru a rotoru BLDC motoru
\medskip
%https://www.st.com/resource/en/product_training/Fundamentals_of_Motor_Control_2020.pdf

Hodnotu točivého momentu mezi statorem a rotorem lze v takovém případě vyjádřit jako

$$ T_{q} = \mid\mbfB_{stat}\mid  \mid\mbfB_{rot}\mid \sin{\muptheta} \eqno (4.4)$$

kde  $\muptheta$ je úhel mezi těmito vektory.

% Elektrotechnika I - Antonín Blahovec vydání paté strana 170-173

Z rovnice 4.3 a 4.4 je zřejmé, že mnoštví a směr procházejícího proudu cívkou ovlivňuje statorové magnetické pole a tím i
směr a velikost točivého momentu motoru. Tento směr a velikost je možné regulovat pomocí řídícího obvodu.

\sec Řídící obvod

Řídící obvod se skládá ze dvou částí. Jedna část zajišťuje tok elektrické energie do motoru a druhá část tento tok řídí.
Obě tyto části jsou navrženy dle požadavků aplikace a mohou se lišit v závislosti na typu zdroje napětí a použitých 
metod řízení. % Electric motors and drives str.45-46

Pro BLDC motory je typické použití stejnosměrného zdroje napětí. 
Hlavními komponentami řídícího obvodu je elektronický komutátor a mikrokontrolér.
Komutátor zajišťuje tok elektrického proudu do motoru a mikrokontrolerér toto množství reguluje.
% permanent magnet motor technology str.270

\sec Elektronický komutátor

% mosfety se spínají rychleji, ale mají vyšší ztráty než IGBT
% ta dioda už je, protože mosfet má integrovanou diodu
% igbt high voltage, high current, low switching frequency
% mosfet low voltage, high current, high switching frequency
% https://www.integrasources.com/blog/bldc-motor-controller-design-principles/


Elektronický komutátor je sestaven z tzv. půlmůstků, přičemž každý půlmůstek se skládá
 ze dvou spínačů\fnote{V případě použití jednoho spínače jsou řídící metody i výkonnost motoru výrazně omezeny.}. 
 Pro tyto účely spínání se dle potřeb využívají tranzistory typu MOSFET nebo IGBT. 
 Každý půlmůstek je připojen k jedné fázi motoru, což umožňuje buď přivedení napětí zdroje 
 na danou fázi, nebo její uzemnění. Tímto způsobem lze přesně řídit tok proudu a směr otáčení motoru.
 % Electric motors and drives str.72-73

Tranzistory spínající napětí zdroje jsou označeny HS (high side) a tranzistory spínající zem jsou označeny jako LS (low side).
Jednotlivé fáze motoru jsou označeny U, V a W. Kombinace těchto dvou označení pak tvoří označení pro jednotlivé tranzistory.
Toto označení není unifikováno, ale v této práci bude použito i v dalších kapitolách. 

% obrázek
\medskip
\picw=12 cm \cinspic Pictures/4rizeni/Bridge.jpg 
\caption/f Řídící můstek třífázového BLDC motoru
\medskip

Pro ochranu tranzistorů před zpětnými proudy z motoru jsou tranzistory osazeny paralelními diodami, které 
umožňují bezpečné odvádění přebytečné energie z motoru.

Elektronický komutátor bývá nejčastěji integrovaný již v řídícím obvodu společně s ochrannými obvody a mikrokontrolerem 
generující signály pro spínání jednotlivých tranzistorů. Řídící obvod pak bývá označován jako ESC\fnote{Electronic Speed Controller}.