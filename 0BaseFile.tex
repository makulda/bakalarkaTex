% The documentation of the usage of CTUstyle -- the template for
% typessetting thesis by plain\TeX at CTU in Prague
% ---------------------------------------------------------------------
% Petr Olsak  Jan. 2013

% You can copy this file to your own file and do some changes.
% Then you can run:  optex your-file

\input ctustyle3  % The template (in version 3, for OpTeX) is included here.

\worktype [B/CZ] % Type: B = bachelor, M = master, D = Ph.D., O = other
                 % / the language: CZ = Czech, SK = Slovak, EN = English

\faculty    {F3}  % Type your faculty F1, F2, F3, etc. or MUVS
            % use main language of your document here:
\department {Katedra měření}
\title      {Řízení BLDC motoru s Hallovými sondami pomocí six-step algoritmu v proudovém režimu }
%\subtitle   {Řízení BLDC motoru s Hallovými sondami pomocí six-step algoritmu v proudovém režimu }
            % \subtitle is optional
\author     {Matouš Kulich}
\date       {Leden 2024}
\supervisor {Ing. Jan Stejskal}  % One or more supervisors
\studyinfo  {Kybernetika a robotika}  % Study programme etc.
%\workname   {Dokumentace} % Used only if \worktype [O/*] (Other)
            % optional more information about the document:
%\workinfo   {\url{http://petr.olsak.net/ctustyle.html}}
            % Title / Subtitle in minor language:
\titleEN    {CTUstyle -- the user manual}
\subtitleEN {the \OpTeX/ template for theses at CTU}
            % If minor language is other than English
            % use \titleCZ, \subtitleCZ or \titleSK, \subtitleSK instead it.
\pagetwo    {}  % The text printed on the page 2 at the bottom.

\abstractEN {
   This document shows and tests an usage of the plain\TeX{} officially
   (may be) recommended design style {\ssr CTUstyle} for bachelor (Bsc.), master
   (Ing.), or doctoral (Ph.D.) theses at the Czech Technical University in
   Prague. The template defines all thesis mandatory structural elements and
   typesets their content to fulfil the university formal rules.

   This is version 3 of this template which is derived from previous version 2 (for
   plain \TeX), but the version 3 supports \OpTeX/ format.
   It implements the Technika font
   recommended by CTU graphics identity reference since 2016.
}
\abstractCZ {
   Tato bakalářská práce se zabívá popisem, modelováním a realizací řízení BLDC motoru. 
}           % If your language is Slovak use \abstractSK instead \abstractCZ

\keywordsEN {%
   document design template; bachelor, master, Ph.D. thesis; \TeX{}.
}
\keywordsCZ {%
   BLDC, motor, řízení, hallova sonda, STEVAL-SPIN3202, STM32
}
\thanks {           % Use main language here
   Chtěl bych poděkovat své manželce Ludmile za podporu nejen finanční.
   Díky tomu mohu na svém pracovišti dělat, co mě baví, a nejsem stresován 
   výplatní páskou.
}
\declaration {      % Use main language here
   Prohlašuji, že jsem předloženou práci vypracoval
   samostatně a že jsem uvedl veškeré použité informační zdroje v~souladu
   s~Metodickým pokynem o~dodržování etických principů při přípravě
   vysokoškolských závěrečných prací.

   V Praze dne 13. 13. 2013 % !!! Attention, you have to change this item.
   \signature % makes dots
}

%%%%% <--   % The place for your own macros is here.

%\draft     % Uncomment this if the version of your document is working only.
%\linespacing=1.7  % uncomment this if you need more spaces between lines
                   % Warning: this works only when \draft is activated!
%\savetoner        % Turns off the lightBlue backround of tables and
                   % verbatims, only for \draft version.
%\blackwhite       % Use this if you need really Black+White thesis.
%\onesideprinting  % Use this if you really don't use duplex printing. 

\makefront  % Mandatory command. Makes title page, acknowledgment, contents etc.

\input 1uvod    % Files where the source of the document is prepared.
\input 2popis   % Full name is: uvod.tex, popis.tex, the suffix can be omitted.
\input 9prilohy

\bye
