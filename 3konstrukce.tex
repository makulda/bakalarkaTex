
\catcode`<=13
\def<#1>{\hbox{$\langle$\it#1\/$\rangle$}}
\toksapp\everytt{\catcode`<=13} \toksapp\everyintt{\catcode`<=13}

\iffalse
Zajímavé fakta a tipy na obsah
Book: Electric Motor Control: CD AC and BLDC
	- DC motor handles about 2000 - 3000 hours (389)
	- BLCD was developed in 1962 (389)
	- BLDC has winding on STATOR and magnets on ROTOR
	- advantages low-cost, high-speed, simple drive method (389)
	- windings -> 1-phase (rotates only in one direction, no self startup), 2-phase, 3-phase and multi-phase (393)
	- 2 type of design -> radial-flux / axial-flux type (393)
		- radial-flux
			- advantages of inner rotor: higher heat dissipating capacity, high torque-to-inertia ration, lower rotor inertia, lower vibration and noise
				- common application is servo
			- advantages of outer rotor: more magnets -> more flux 
				- common application: app with const speed, computer disk drives, cooling fans
		- axial-flux
			- advantages of axial type: slim structure, shorter axial length
				- common application: electical vehicles in-wheel motors, elevator motors

	- Stator a rotor
	- Vinutí
	- Permanentní magnety
	- Komutátor
	- řídící elektronika
	- chlazení
	- konstrukční materiály?

\fi
\chap Konstrukce BLDC motoru
% uvodní slovo do kapitoly
BLDC motory nabízí širokou škálu konstrukčních variant a právě proto jsou populární volbou v mnoha aplikacích. 
V této kapitole se budeme věnovat detailnímu popisu konstrukčních prvků BLDC motorů, jejich variantám a
 elektronickým komponentám pro správný chod motoru. Začneme popisem statoru a rotoru, které jsou základními 
 konstrukčními prvky motoru. Dále se zaměříme na vinutí fází a využití permanentních magnetů. 
 Následně se seznámíme s technikami chlazení a nakonec se zaměříme na řídící elektroniku, která 
 umožňuje efektivní řízení otáček a točivého momentu motoru. 

%BLDC motory představují moderní kategorii elektromotorů, které nabízí širokou škálu konstrukčních variant.
%Konstrukční variabilita BLDC motorů je závislá na požadavcích konkrétní aplikace, ve které bude motor využit.
%Nejbějží konstrukce jsou s radiálním a axiálním uspořádáním. Tyto konstrukce se liší umístěním rotoru vůči statoru a 
%mají tak různé vlastnosti. Pojdmě si nyní vysvětlit, co je to stator a rotor.

\sec Stator a rotor

Stator a rotor jsou základním konstrukční prvkem každého elektromotoru. Stator, často označovaný jako neměnná nebo nepohyblivá část, obsahuje vinuté cívky, kterými prochází elektrický
 proud generující magnetické pole. Naopak rotor představuje pohyblivou část motoru, která koná rotační pohyb kolem statoru a přenáší tak točivý moment na hřídel či na objekt, s kterým chceme pohybovat. 
 Rotor může být osazen permanentními magnety nebo vinutými cívkami. V takovém případě hovoříme o cize buzeném motoru. V případě BLDC motorů je nejběžnější použití permanentních magnetů a proto se 
 zaměříme právě na tyto konstrukční řešení. 

Setkáváme se s několika konstrukčními variantami, které se liší umístěním rotoru vůči statoru. Mužeme je rozdělit na radiální a axiální uspořádání.
Dále dále můžeme stator rozdělit na tzv. slot a slotless variantu. Pojdmě si nyní tyto konstrukční varianty detailněji představit.   
\secc Radiální uspořádání
S radiálním uspořádáním se BLDC motorů setkáváme nejčastěji. V tomto uspořádání je rotor umístěn buď uvnitř, nebo vně statoru. 
V obou případech je stator a rotor oddělen vzduchovou mezerou, která zajišťuje minimální mechanický kontakt mezi oběma částmi. 
Takové uspořádaní si můžeme představit jako dvě koncentrické trubice, kde jedna trubice představuje stator a druhá rotor.
Příklad BLDC motoru s radálním uspořádáním můžeme vidět na obrázku 3.1 níže.

\medskip
\picw=.5\hsize
\clabel[obr.a+obr.b]{Popisek k obrázkům} % to dá popisek do seznamu obrázků
\centerline {\inspic Pictures/3konstrukce/in-complet.png \hfil\hfil \inspic Pictures/3konstrukce/out-complet.png }\nobreak
\centerline {a)\hfil\hfil b)}\nobreak\medskip
\caption/f Radiální BLDC motor a) se statorem uvnitř rotoru b) s rotorem uvnitř statoru.
 \medskip
 
Ačkoli jde o stejný typ motoru, jejich klíčové vlastnosti jsou rozdílné a každý z nich je tak vhodný pro jiné aplikace. 
. Tyto rozdíly a typické vlastnosti si popíšeme v kapitole % TODO: napsat kapitolu. 

\secc Axiální uspořádání
Při axiálním upsořádání je rotor a stator umístěn proti sobě. Osa rotace rotoru a statoru je pak kolmá k těmto plochám.
Příklad jednoduchého BLDC motoru s axiálním uspořádáním můžeme vidět na obrázku 3.2 níže. 

 \medskip
 \picw=9 cm \cinspic Pictures/3konstrukce/axial-complet.png 
 \centerline {c)}\nobreak\medskip
 \caption/f Ukázka vložení obrázku na střed, což je asi nejobvyklejší.
 \medskip
 
 Axiální uspořádání nabízí i několik složitějších konstrukčních variant, kdy je rotor resp. stator zastoupen dvěma kusy.
 Dvojitý rotor nebo stator zvyšuje výkon motoru s minimálním nárokem na zvětšení jeho rozměrů. Avšak s rostoucí komplexností konstrukce
 ztrácí motor na své jednoduchosti, kterou se BLDC motory vyznačují a pro které jsou tak oblíbené.     

\secc Konstrukční materiály
BLDC motory jsou obvykle vyrobeny z feromagnetických materiálů, jako je 
například ocel a další kovové slitiny, které poskytují dostatečnou magnetickou propustnost a stabilitu 
pro správnou funkci motoru.

 \secc Slot vs. Slotless konstrukce
 Pojmem slot se označují drážky statoru, kolem kterých jsou umístěné vinuté cívky. Tyto drážky jsou zvýrazněny modrou barvou
 na obrázku 3.3 níže. 

 \medskip
 \picw=7 cm \cinspic Pictures/3konstrukce/slot.png 
 \caption/f Sloty BLDC motoru.
 \medskip

 Drážky umožňují snadné upevnění cívek a minimalizují riziko jejich pohybu nebo poškození během provozu motoru. 
 Další výhodou drážek je zvýšení a usměrnění magnetického toku generovaný vinutými cívkami.   
 Hlavní nevýhodou slot konstrukce je interakce mezi drážkami a permanentními magnety, které jsou umístěny na rotoru. 
 Tato interakce způsobuje tzv. {\em cogging torque}, což je nežádoucí točivý moment motoru, který se negativně projeví
  v plynulosti chodu motoru. Právě tato negativní vlastnost drážkových statorů vedla k vývoji slotless konstrukce, která
 minimalizuje vliv drážek na chování motoru. Cívky jsou u slotless motorů umístěny přímo na povrchu statoru a pro jejich fixaci se využívají
 speciální lepidla. Vinutí cívek obou typů motorů je zobrazeno v kapitole Vinutí fází.

 % společný prvek

 \sec Vinutí fází
- materiál (izolace)
- zapojení (trojúhelníkové, hvězdicové)
- počet cívek
- cogling torque 

 \sec Permanentní magnety
 - strana 391 BLDC knihy ->

- materiál
- tvar
- umístění

\sec Porovnání zmíněných konstrukcí

 \sec Komutátor

 \sec Řídící elektronika
 - ESC (Electronic Speed Control)
 \sec Chlazení

 