
\catcode`<=13
\def<#1>{\hbox{$\langle$\it#1\/$\rangle$}}
\toksapp\everytt{\catcode`<=13} \toksapp\everyintt{\catcode`<=13}

\iffalse
Zajímavé fakta a tipy na obsah
Book: Electric Motor Control: CD AC and BLDC
	- DC motor handles about 2000 - 3000 hours (389)
	- BLCD was developed in 1962 (389)
	- BLDC has winding on STATOR and magnets on ROTOR
	- advantages low-cost, high-speed, simple drive method (389)
	- windings -> 1-phase (rotates only in one direction, no self startup), 2-phase, 3-phase and multi-phase (393)
	- 2 type of design -> radial-flux / axial-flux type (393)
		- radial-flux
			- advantages of inner rotor: higher heat dissipating capacity, high torque-to-inertia ration, lower rotor inertia, lower vibration and noise
				- common application is servo
			- advantages of outer rotor: more magnets -> more flux 
				- common application: app with const speed, computer disk drives, cooling fans
		- axial-flux
			- advantages of axial type: slim structure, shorter axial length
				- common application: electical vehicles in-wheel motors, elevator motors

	- Stator a rotor
	- Vinutí
	- Permanentní magnety
	- Komutátor
	- řídící elektronika
	- chlazení
	- konstrukční materiály?

\fi
\chap Konstrukce BLDC motoru
% uvodní slovo do kapitoly
BLDC motory nabízí širokou škálu konstrukčních variant a právě proto jsou populární volbou v mnoha aplikacích. 
V této kapitole se budeme věnovat detailnímu popisu konstrukčních prvků BLDC motorů, jejich variantám a
 elektronickým komponentám pro správný chod motoru. Začneme popisem statoru a rotoru, které jsou základními 
 konstrukčními prvky motoru. Dále se zaměříme na vinutí fází a využití permanentních magnetů. 
 Následně se seznámíme s technikami chlazení a nakonec se zaměříme na řídící elektroniku, která umožňuje efektivní řízení otáček a momentu motoru. 

%BLDC motory představují moderní kategorii elektromotorů, které nabízí širokou škálu konstrukčních variant.
%Konstrukční variabilita BLDC motorů je závislá na požadavcích konkrétní aplikace, ve které bude motor využit.
%Nejbějží konstrukce jsou s radiálním a axiálním uspořádáním. Tyto konstrukce se liší umístěním rotoru vůči statoru a 
%mají tak různé vlastnosti. Pojdmě si nyní vysvětlit, co je to stator a rotor.

\sec Stator a rotor
radial-flux / axial-flux
Stator, často označovaný jako neměnná nebo nepohyblivá část. Obsahuje vinuté cívky, kterými prochází elektrický
 proud generující magnetické pole. Naopak rotor představuje pohyblivou část motoru, která koná rotační pohyb kolem statoru.

Vnitřní rotor je vhodný pro aplikace, kde je vyžadována vysoká tepelná odolnost, vysoký poměr točivého momentu k setrvačnosti, 
nízká setrvačnost rotoru a nízká hlučnost. 

Naopak vnější rotor je vhodný pro aplikace, kde je vyžadován vysoký točivý moment,  
Další výhodou vnějšího rotoru je možnost umístit více magnetů, což zvyšuje magnetický tok a tím i výkon motoru.
V některých aplikacích je takové řešení 

Axiální uspořádání je vhodné pro aplikace, kde je vyžadována kompaktní konstrukce a krátká délka motoru.
Tento typ konstrukce je vhodný pro aplikace, kde je omezený prostor, jako například elektromobily, výtahy nebo chlazení.


 V rámci této kapitoly se detailněji zaměříme na konstrukční řešení a nezbytné komponenty pro chod motoru.
 
 \medskip
\picw=.6\hsize
\clabel[obr.a+obr.b]{Popisek k obrázkům} % to dá popisek do seznamu obrázků
\centerline {\inspic Pictures/3konstrukce/in-rotor-stator.png \hfil\hfil \inspic Pictures/3konstrukce/out-rotor-stator.png }\nobreak
\centerline {a)\hfil\hfil b)}\nobreak\medskip
\caption/f Radiální BLDC motor a) se statorem uvnitř rotoru b) s rotorem uvnitř statoru.
 \medskip

 \medskip
 \picw=9 cm \cinspic Pictures/3konstrukce/axial-rotor-stator.png 
 \caption/f Ukázka vložení obrázku na střed, což je asi nejobvyklejší.
 \medskip
 
 - axialní
 - radiální
    - rotor in / out

 - coreless (slotless) motor
	- výhody
	- nevýhody
	- aplikace	


	- motory s rotorem uvnitř statoru mají menší torque -> jsou delší než s rotorem out
 \cite[druha]
 % společný prvek

 \sec Vinutí fází
- materiál (izolace)
- zapojení (trojúhelníkové, hvězdicové)
- počet cívek
- cogling torque 

\medskip
\picw=.6\hsize
\clabel[obr.1+obr.2]{Popisek} % to dá popisek do seznamu obrázků
\centerline {\inspic Pictures/3konstrukce/in-coil.png \hfil\hfil \inspic Pictures/3konstrukce/out-coil.png }\nobreak
\centerline {a)\hfil\hfil b)}\nobreak\medskip
\caption/f Radiální BLDC motor a) se statorem uvnitř rotoru b) s rotorem uvnitř statoru.
 \medskip
Mezera 
 \medskip
\picw=9 cm \cinspic Pictures/3konstrukce/axial-coil.png 
\caption/f Ukázka vložení obrázku na střed, což je asi nejobvyklejší.
\medskip


 \sec Permanentní magnety

 \medskip
\picw=.6\hsize
\clabel[obr.3+obr.4]{Popisek} % to dá popisek do seznamu obrázků
\centerline {\inspic Pictures/3konstrukce/in-complet.png \hfil\hfil \inspic Pictures/3konstrukce/out-complet.png }\nobreak
\centerline {a)\hfil\hfil b)}\nobreak\medskip
\caption/f Radiální BLDC motor a) se statorem uvnitř rotoru b) s rotorem uvnitř statoru.
 \medskip
Mezera 
 \medskip
\picw=9 cm \cinspic Pictures/3konstrukce/axial-complet.png 
\caption/f Ukázka vložení obrázku na střed, což je asi nejobvyklejší.
\medskip

- materiál
- tvar
- umístění

 \sec Komutátor

 \sec Řídící elektronika
 - ESC (Electronic Speed Control)
 \sec Chlazení

 