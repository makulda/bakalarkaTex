
%Zajímavé fakta a tipy na obsah
%Book: Electric Motor Control: CD AC and BLDC
%	- DC motor handles about 2000 - 3000 hours (389)
%	- BLCD was developed in 1962 (389)
%	- BLDC has winding on STATOR and magnets on ROTOR
%	- advantages low-cost, high-speed, simple drive method (389)
%	- windings -> 1-phase (rotates only in one direction, no self startup), 2-phase, 3-phase and multi-phase (393)
%	- 2 type of design -> radial-flux / axial-flux type (393)
%		- radial-flux
%			- advantages of inner rotor: higher heat dissipating capacity, high torque-to-inertia ration, lower rotor inertia, lower vibration and noise
%				- common application is servo
%			- advantages of outer rotor: more magnets -> more flux 
%				- common application: app with const speed, computer disk drives, cooling fans
%		- axial-flux
%			- advantages of axial type: slim structure, shorter axial length
%				- common application: electical vehicles in-wheel motors, elevator motors
%
%	- Stator a rotor
%	- Vinutí
%	- Permanentní magnety
%	- Komutátor
%	- řídící elektronika
%	- chlazení
%	- konstrukční materiály?
%
%\fi

\chap Konstrukce BLDC motoru

% uvodní slovo do kapitoly
BLDC motory disponují rozmanitými konstrukčními variantami, díky kterým jsou vhodné pro široké spektrum aplikací.
Tato kapitola se zabývá detailním popisem konstrukčních prvků BLDC motorů a jejich variant.

%BLDC motory představují moderní kategorii elektromotorů, které nabízí širokou škálu konstrukčních variant.
%Konstrukční variabilita BLDC motorů je závislá na požadavcích konkrétní aplikace, ve které bude motor využit.
%Nejbějží konstrukce jsou s radiálním a axiálním uspořádáním. Tyto konstrukce se liší umístěním rotoru vůči statoru a 
%mají tak různé vlastnosti. Pojdmě si nyní vysvětlit, co je to stator a rotor.

\sec Stator a rotor

Stator a rotor jsou základním konstrukčním prvkem každého elektromotoru. Stator, často označovaný jako neměnná nebo nepohyblivá část, obsahuje vinuté cívky, kterými prochází elektrický
 proud generující magnetické pole. Naopak rotor představuje pohyblivou část motoru, která koná rotační pohyb kolem statoru a přenáší tak točivý moment na hřídel či na objekt, s kterým chceme pohybovat. 
 Rotor může být osazen permanentními magnety nebo vinutými cívkami. V takovém případě hovoříme o cize buzeném motoru. V případě BLDC motorů je nejběžnější použití permanentních magnetů a proto se 
 zaměříme právě na tato konstrukční řešení. % Brushless D.C. Motors without Permanent Magnets 
% https://www.integratedsoft.com/documents/tech_6mx.pdf

 BLDC motory se v závislosti na umístění rotoru vůči statoru dělí na radiální a axiální uspořádání. 
 Dle konstrukce statoru se dále dělí na slot a slotless konstrukci.

\secc Radiální uspořádání

Radiální uspořádání je nejběžnější konstrukční variantou BLDC motorů. 
V tomto uspořádání je rotor umístěn uvnitř, nebo vně statoru.
V obou případech je stator a rotor oddělen vzduchovou mezerou, která zajišťuje minimální mechanický kontakt mezi oběma částmi. 
Takové uspořádaní si můžeme představit jako dvě koncentrické trubice, kde jedna trubice představuje stator a druhá rotor.
Příklad BLDC motoru s radiálním uspořádáním můžeme vidět na obrázku 3.1 níže.

\medskip
\picw=.35\hsize
\clabel[obr.a+obr.b]{Popisek k obrázkům} % to dá popisek do seznamu obrázků
\centerline {\inspic Pictures/3konstrukce/in-big.png \hfil\hfil \inspic Pictures/3konstrukce/out-big.png }\nobreak
\centerline {a)\hfil\hfil b)}\nobreak\medskip
\caption/f Radiální BLDC motor a) se statorem uvnitř rotoru b) s rotorem uvnitř statoru.
 \medskip
 
 Každý typ má své specifické vlastnosti, které je třeba zvážit při výběru motoru pro konkrétní aplikaci.
  
V uspořádání s vnějším rotorem působí interakce mezi permanentními magnety a statorovými cívkami ve větší vzdálenosti
od osy otáčení a disponují tak vyšším točivým momentem. Vzhledem k této vlastnosti jsou obvykle ve srovnání se svým průměrem kratší.
Jsou tak ideální volbou pro aplikace vyžadující vysoký točivý moment s kompaktními rozměry v podélném směru.
Nacházejí tak uplatnění například jako pohonné jednotky pro ventilátory, drony či automobily.

Naopak motory s uspořádáním rotoru uvnitř statoru disponují díky menšímu průměru rotoru nižším momentem setrvačnosti a menšími vibracemi.
Dosahují tak lepšího dynamického chování a nižší hlučnosti. 
Dále díky uzavřené konstrukci jsou méně náchylné na vnější vlivy jako je prach, voda a jiné nečistoty. 
Jsou rovněž preferovány pro aplikace v blízkosti lidí, neboť neobsahují volně rotující součásti představující riziko úrazu.
Nacházejí tak uplatnění nejen jako pohony v robotice a průmyslové automatizaci, ale i v běžné spotřební elektronice jako 
jsou například vysavače a kuchyňské spotřebiče.

\secc Axiální uspořádání

Při axiálním uspořádání jsou plochy rotoru a statoru umístěny rovnoběžně proti sobě. Osa rotace rotoru je pak kolmá k těmto plochám.
Příklad jednoduchého BLDC motoru s axiálním uspořádáním je znázorněn na obrázku 3.2 níže. 

 \medskip
 \picw=9 cm \cinspic Pictures/3konstrukce/axial-complet.png 
 \caption/f Axiální uspořádání BLDC motoru.
 \medskip

Axiální uspořádání poskytuje i možnost implementace složitějších konstrukčních variant, které zahrnují
použití více částí rotorů a statorů. Takové konstrukce umožňují zvýšení výkonu motoru při minimálním nárůstu jeho rozměrů.

Hlavní předností axiálního uspořádání je vysoký výkon motoru s velmi kompaktními rozměry. Jsou tak hojně 
využívány v aplikacích, kdy jsou již zabudované do rotujících částí. Typickými příklady takových aplikací jsou 
pohony pro jízdní kola či automobily, kde jsou umístěny uvnitř kol. 
% Axial Flux Permanent Magnet Brushless Machines pp 281–325

\secc Konstrukční materiály

BLDC motory jsou obvykle vyrobeny z feromagnetických materiálů, jako je 
například ocel a další kovové slitiny, které poskytují dostatečnou magnetickou propustnost a stabilitu 
pro správnou funkci motoru.

 \secc Slot vs. slotless konstrukce

 Pojmem slot se označují drážky statoru, kolem kterých jsou umístěné vinuté cívky. Tyto drážky jsou zvýrazněny modrou barvou
 na obrázku 3.3 níže.

 \medskip
 \picw=7 cm \cinspic Pictures/3konstrukce/slot.png 
 \caption/f Sloty BLDC motoru
 \medskip

 Drážky umožňují snadné upevnění cívek a minimalizují riziko jejich pohybu nebo poškození během provozu motoru. 
 Další výhodou drážek je přispění k rovnoměrnému usměrnění magnetického toku generovaný vinutými cívkami.
 Dále také přispívají k efektivnějšímu odvodu tepla z cívek a zlepšují tak tepelné vlastnosti motoru.

 Hlavní nevýhodou slot konstrukce je nežádoucí interakce mezi drážkami a permanentními magnety rotoru. 
 Tato interakce způsobuje tzv. {\em cogging torque}, což je nežádoucí točivý moment motoru. % Analysis of Back EMF Harmonics Influenced by Slot-Pole Combinations in Permanent Magnet Vernier In-Wheel Motors
 Ten se negativně projeví v plynulosti chodu motoru. Právě tato negativní vlastnost drážkových statorů je u bezdrážkových statorů eliminována.
  Cívky slotless motorů jsou umístěny přímo na povrchu statoru a pro jejich fixaci se využívají
 speciální lepidla. % Permanent motor technology str.123 - 124

 Slotless motory jsou využívany v lékařských přístrojích vyžadujících vysokou přesnost a plynulost chodu. Pro stejné účely jsou 
 využívány i ve vojenském a kosmickém průmyslu.

% Permanent motor technology
 % společný prvek

 \sec Cívky BLDC motoru

 Cívky do elektromotrů jsou nejčastěji vyrobeny z mědi nebo hliníku a jsou vždy opatřeny povrchovou izolací, 
 která zabraňuje vzniku zkratu mezi jednotlivými vinutími.
 Tato izolace je obvykle zajištená pomocí speciálních izolačních laků a jejich složení je závislé na konkrétní aplikaci motoru.
 Základní složkou jsou organické pryskyřice, které zajišťují izolační vlastnosti a odolnost proti vysokým teplotám. Další přidané složky
 mohou zvyšovat mechanickou pružnost, odolnost proti chemickým látkám i ochraně proti UV záření. % Technologie vinutí elektrických strojů točivých 13-15, 44

Cívky jsou pak navinuté na stator motoru do tzv. koncentrovaného vinutí. 
Tedy takovým způsobem, že se jednotlivé fáze nepřekrývají a jsou odděleny vzduchovou mezerou.

BLDC motory se vyznačují několika typy vinutí, které se liší počtem fází. Při použití tří a více fází 
se fáze zapojují do tzv. hvězdy nebo do trojúhelníku. 

\secc Počet fází

BLDC motory mohou mít 1-fázové, 2-fázové, 3-fázové a multi-fázové vinutí.
S příbajícím počtem fází se zlepšují jeho vlastnosti, ale také jeho cena a složitost řízení. 
Proto je důležité zvolit takový počet fází, který bude vyhovovat požadavkům konkrétní aplikace.

Jednofázové motory jsou nejjednodušším typem BLDC motoru jak z pohledu konstrukce tak i z pohledu řízení. 
Disponují však účinností pouze okolo 50 \% a nízkým točivým momentem.
 Další nevýhodou je veliké kolísání točivého momentu a možnost rotace pouze v jednom směru.
 Tyto vlastnosti jsou ale dostačující pro aplikace jako pohony pro jednoduché ventilátory a čerpadla.
% https://ieeexplore-ieee-org.ezproxy.techlib.cz/stamp/stamp.jsp?tp=&arnumber=5663588

Dvoufázové motory nabízí vyšší točivý moment a možnost rotace v obou směrech.
Nabízí i pokročilejší a efektivnější řídící metody.
Při použití jednoduchých řídících metod však stále trpí značným kolísáním točivého momentu.
Své uplatnění nacházejí v kuchyňských spotřebičích, ventilátorech a čerpadlech. 

% účinnost pro dvoufázové motory se mi nepodařilo dohledat

Třífázové motory jsou nejběžnějším typem BLDC motorů. Z tohoto důvodu se tato práce zaměřuje právě na tyto motory.
Oproti předchozím typům nabízejí efektivnější řízení a to jak pro jednoduché tak i pokročilejší metody.
Třífázové motory se hojně využívají v průmyslové automatizaci, letectví, robotice a v elektromobilních aplikacích. 

Multi-fázové motory jsou nejvýkonnější volbou a oproti předchozím typům nabízí vyšší spolehlivost. 
Právě díky více fázím je schopen pracovat s dostatečnou účinností i při poškození jedné či více fází. 
Tento typ motoru se využívá v odvětvích jako je například letectví, kosmonautika či ve vojenském průmyslu. 
% Design and Analysis of Multi-Phase BLDC Motors for Electric Vehicles 
% Comparing the Performance of Parallel Multi-Phase Brushless DC Motors: A Comprehensive Analysis

\secc Zapojení vinutí do trojúhelníku

%- méně časté (circulating current -> Při pohybu rotoru BLDC motoru se mění magnetický tok v cívkách a indukuje se napětí. Pokud není tento jev správně kontrolován, může indukované napětí způsobit proudění proudu mezi fázemi motoru. Tento oběhový proud může způsobit ztráty výkonu a zahřívání motoru, což může vést k poklesu účinnosti a životnosti motoru. Aby byl oběhový proud minimalizován nebo eliminován, je důležité provést vhodný návrh motoru a řídicího systému. To zahrnuje optimalizaci geometrie motoru, volbu správného typu cívek a magnetického obvodu, a implementaci pokročilých řídicích algoritmů, které minimalizují indukované napětí a řídí proudy motoru tak, aby byly co nejefektivnější. Takové opatření pomáhají snižovat ztráty výkonu a zvyšovat účinnost BLDC motorů.)
%- vyšší počáteční proud (vyšší počáteční točivý moment)
%- obtížnější detekce polohy rotoru 

Zapojení vinutí do trojúhelníku se u BLDC motorů využívá jen zřídka. Kvůli uzavřenému obvodu může docházet k tzv. oběhovému proudu.
Ten vzniká při změně magnetického toku v cívkách v důsledku pohybu rotoru, což indukuje napětí a způsobuje tak
proud mezi fázemi motoru. Tento jev může způsobit zahřívání a ztráty výkonu motoru, což negativně ovlivňuje jeho účinnost a životnost. 
Pro potlačení tohoto jevu je nutné provést optimalizovaný návrh motoru a řídicího systému. 
%Dalším důsledkem uzavřeného obvodu je obtížnější detekce polohy rotoru.
%V trojúhelníkové zapojení nastává 30° fázový posun mezi svorkovým a fázovým napětím. Je tedy nezbytné 
%při detekci polohy rotoru s touto informací pracovat. Opět tak dochází k vyšším nárokům na návrh motoru.
Jedinou výhodou tohoto zapojení je vyšší počáteční proud při rozběhu motoru. Důsledkem toho motor disponuje 
vyšším počátečním točivým momentem, který může být v některých aplikacích žádoucí. 

\medskip
\picw=.35\hsize
\clabel[vinuti-trojuhelnik]{Popisek k obrázkům} % to dá popisek do seznamu obrázků
\centerline {\inspic Pictures/3konstrukce/sign-delta.png  \hfil\hfil \inspic Pictures/3konstrukce/coil-delta.png  }\nobreak
%\centerline {a)\hfil\hfil b)}\nobreak\medskip
\caption/f Vinutí cívek do trojúhelníku ve schématu a při realizaci
 \medskip

\secc Zapojení vinutí do hvězdy

Zapojení vinutí do hvězdy je realizováno spojením jednotlivých fází do jednoho tzv. neutrálního bodu.
V důsledku zapojení jednotlivých fází do série se odpory cívek sčítají a proudy protékající cívkami jsou
menší než u trojúhelníkového zapojení. Počáteční proudové špičky jsou tak menší a motor je tak méně zatěžován.
Důsledkem je také menší počáteční točivý moment, který je však stále dostatečný pro většinu aplikací.

\medskip
\picw=.35\hsize
\clabel[vinuti-hvezda]{Popisek k obrázkům} % to dá popisek do seznamu obrázků
\centerline {\inspic Pictures/3konstrukce/sign-star.png  \hfil\hfil \inspic Pictures/3konstrukce/coil-star-n.png  }\nobreak
%\centerline {a)\hfil\hfil b)}\nobreak\medskip
\caption/f Vinutí cívek do hvězdy ve schématu a při realizaci
 \medskip

Obvod je oproti trojúhelníkovému zapojení otevřený a nevzniká tak oběhový proud. Díky tomu jsou řídící metody přesnější a efektivnější.

Při použití jednoduchých řídících metod jsou aktivní pouze dvě fáze. Třetí fáze je plavoucí\fnote{na fázi není přivedeno kladné napětí ani není uzemněna} a 
lze díky ní detekovat polohu rotoru. Tato metoda u zapojení do trojúhelníku nelze použít.
%https://www.researchgate.net/publication/338582249_Design_Code_Generation_and_Simulation_of_a_BLDC_Motor_Controller_usuuing_PIC_Microcontroller

Tyto vlastnosti činí zapojení vinutí do hvězdy preferovanou volbou pro BLDC motory.

 \sec Permanentní magnety

Permanentní magnety jsou umístěny na/v rotoru motoru a generují magnetické pole, které interaguje s magnetickým polem statoru.
Dle počtu magnetů v motoru se určí počet tzv. pól-párů. Každý pól-pár je tvořen dvěma permanentními magnety usazenými v rotoru v opačné polaritě.

Pro BLDC motory se využívají především neodymové, feritové a samario-kobaltové magnety. Každý z uvedených typů magnetů 
disponuje odlišnými vlastnostmi. Proto je důležité zvolit takový typ magnetu, který bude vyhovovat požadavkům konkrétní aplikace.

Pro velmi výkonné a účinné motory se používají neodymové magnety, které disponují nejvyšší energií na jednotku objemu (1,2 - 1,4 T). Jejich nevýhodou je však
 náchylnost k odmagnetování a ke korozi. Jsou také stabilní pouze při teplotách do 80 - 250 °C v závislosti na třídě. Jsou tak využívany 
 např. v průmyslovém odvětví, kde lze zajistit ideální prostředí pro jejich provoz.

 Pro aplikace vyžadující větší odolnost vůči vyšším teplotám, odmagnetování a korozi se používají samario-kobaltové magnety. 
 Tyto magnety dokáží pracovat v teplotách až 350 °C a dosahují magnetické indukce až 0,9 - 1,1 T. 

Feritové magnety jsou ekonomicky nejvýhodnější a jsou tak využívány v levných BLDC motorech pro běžné spotřebitelské aplikace. 
Dokáží pracovat v teplotách až 400~°C a jsou odolné vůči korozi a odmagnetování.
Nevýhodou oproti výše zmíněným typům je nižší energie na jednotku objemu (0,4 - 0,5 T).

%https://link-springer-com.ezproxy.techlib.cz/article/10.1007/s42835-020-00397-7
%Permanent motor technology


 