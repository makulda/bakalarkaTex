
%Zajímavé fakta a tipy na obsah
%Book: Electric Motor Control: CD AC and BLDC
%	- DC motor handles about 2000 - 3000 hours (389)
%	- BLCD was developed in 1962 (389)
%	- BLDC has winding on STATOR and magnets on ROTOR
%	- advantages low-cost, high-speed, simple drive method (389)
%	- windings -> 1-phase (rotates only in one direction, no self startup), 2-phase, 3-phase and multi-phase (393)
%	- 2 type of design -> radial-flux / axial-flux type (393)
%		- radial-flux
%			- advantages of inner rotor: higher heat dissipating capacity, high torque-to-inertia ration, lower rotor inertia, lower vibration and noise
%				- common application is servo
%			- advantages of outer rotor: more magnets -> more flux 
%				- common application: app with const speed, computer disk drives, cooling fans
%		- axial-flux
%			- advantages of axial type: slim structure, shorter axial length
%				- common application: electical vehicles in-wheel motors, elevator motors
%
%	- Stator a rotor
%	- Vinutí
%	- Permanentní magnety
%	- Komutátor
%	- řídící elektronika
%	- chlazení
%	- konstrukční materiály?
%
%\fi

\chap Konstrukce BLDC motoru

% uvodní slovo do kapitoly
BLDC motory nabízí širokou škálu konstrukčních variant a právě proto jsou populární volbou v mnoha aplikacích. 
V této kapitole se budeme věnovat detailnímu popisu konstrukčních prvků BLDC motorů, jejich variantám a
 elektronickým komponentám pro správný chod motoru. Začneme popisem statoru a rotoru, které jsou základními 
 konstrukčními prvky motoru. Dále se zaměříme na vinutí fází a využití permanentních magnetů. 
 Následně se seznámíme s technikami chlazení a nakonec se zaměříme na řídící elektroniku, která 
 umožňuje efektivní řízení otáček a točivého momentu motoru. 

%BLDC motory představují moderní kategorii elektromotorů, které nabízí širokou škálu konstrukčních variant.
%Konstrukční variabilita BLDC motorů je závislá na požadavcích konkrétní aplikace, ve které bude motor využit.
%Nejbějží konstrukce jsou s radiálním a axiálním uspořádáním. Tyto konstrukce se liší umístěním rotoru vůči statoru a 
%mají tak různé vlastnosti. Pojdmě si nyní vysvětlit, co je to stator a rotor.

\sec Stator a rotor

Stator a rotor jsou základním konstrukční prvkem každého elektromotoru. Stator, často označovaný jako neměnná nebo nepohyblivá část, obsahuje vinuté cívky, kterými prochází elektrický
 proud generující magnetické pole. Naopak rotor představuje pohyblivou část motoru, která koná rotační pohyb kolem statoru a přenáší tak točivý moment na hřídel či na objekt, s kterým chceme pohybovat. 
 Rotor může být osazen permanentními magnety nebo vinutými cívkami. V takovém případě hovoříme o cize buzeném motoru. V případě BLDC motorů je nejběžnější použití permanentních magnetů a proto se 
 zaměříme právě na tyto konstrukční řešení. 

Setkáváme se s několika konstrukčními variantami, které se liší umístěním rotoru vůči statoru. Mužeme je rozdělit na radiální a axiální uspořádání.
Dále dále můžeme stator rozdělit na tzv. slot a slotless variantu. Pojdmě si nyní tyto konstrukční varianty detailněji představit.   
\secc Radiální uspořádání

S radiálním uspořádáním se BLDC motorů setkáváme nejčastěji. V tomto uspořádání je rotor umístěn buď uvnitř, nebo vně statoru. 
V obou případech je stator a rotor oddělen vzduchovou mezerou, která zajišťuje minimální mechanický kontakt mezi oběma částmi. 
Takové uspořádaní si můžeme představit jako dvě koncentrické trubice, kde jedna trubice představuje stator a druhá rotor.
Příklad BLDC motoru s radálním uspořádáním můžeme vidět na obrázku 3.1 níže.

\medskip
\picw=.35\hsize
\clabel[obr.a+obr.b]{Popisek k obrázkům} % to dá popisek do seznamu obrázků
\centerline {\inspic Pictures/3konstrukce/in-big.png \hfil\hfil \inspic Pictures/3konstrukce/out-big.png }\nobreak
\centerline {a)\hfil\hfil b)}\nobreak\medskip
\caption/f Radiální BLDC motor a) se statorem uvnitř rotoru b) s rotorem uvnitř statoru.
 \medskip
 
Ačkoli jde o stejný typ motoru, jejich klíčové vlastnosti jsou rozdílné a každý z nich je tak vhodný pro jiné aplikace. 
. Tyto rozdíly a typické vlastnosti si popíšeme v kapitole % TODO: napsat kapitolu. 

\secc Axiální uspořádání

Při axiálním upsořádání je rotor a stator umístěn proti sobě. Osa rotace rotoru a statoru je pak kolmá k těmto plochám.
Příklad jednoduchého BLDC motoru s axiálním uspořádáním můžeme vidět na obrázku 3.2 níže. 

 \medskip
 \picw=9 cm \cinspic Pictures/3konstrukce/axial-complet.png 
 \centerline {c)}\nobreak\medskip
 \caption/f Ukázka vložení obrázku na střed, což je asi nejobvyklejší.
 \medskip
 
 Axiální uspořádání nabízí i několik složitějších konstrukčních variant, kdy je rotor resp. stator zastoupen dvěma kusy.
 Dvojitý rotor nebo stator zvyšuje výkon motoru s minimálním nárokem na zvětšení jeho rozměrů. Avšak s rostoucí komplexností konstrukce
 ztrácí motor na své jednoduchosti, kterou se BLDC motory vyznačují a pro které jsou tak oblíbené.     

\secc Konstrukční materiály

BLDC motory jsou obvykle vyrobeny z feromagnetických materiálů, jako je 
například ocel a další kovové slitiny, které poskytují dostatečnou magnetickou propustnost a stabilitu 
pro správnou funkci motoru.

 \secc Slot vs. Slotless konstrukce

 Pojmem slot se označují drážky statoru, kolem kterých jsou umístěné vinuté cívky. Tyto drážky jsou zvýrazněny modrou barvou
 na obrázku 3.3 níže.

 \medskip
 \picw=7 cm \cinspic Pictures/3konstrukce/slot.png 
 \caption/f Sloty BLDC motoru.
 \medskip

 Drážky umožňují snadné upevnění cívek a minimalizují riziko jejich pohybu nebo poškození během provozu motoru. 
 Další výhodou drážek je zvýšení a usměrnění magnetického toku generovaný vinutými cívkami.   
 Hlavní nevýhodou slot konstrukce je interakce mezi drážkami a permanentními magnety, které jsou umístěny na rotoru. 
 Tato interakce způsobuje tzv. {\em cogging torque}, což je nežádoucí točivý moment motoru, který se negativně projeví
  v plynulosti chodu motoru. Právě tato negativní vlastnost drážkových statorů vedla k vývoji slotless konstrukce, která
 minimalizuje vliv drážek na chování motoru. Cívky jsou u slotless motorů umístěny přímo na povrchu statoru a pro jejich fixaci se využívají
 speciální lepidla. Vinutí cívek obou typů motorů je zobrazeno v kapitole Vinutí fází.

 % společný prvek

 \sec Cívky BLDC motoru

 Cívky do elektromotrů jsou nejčastěji vyrobeny z mědi, nebo hliníku a jsou vždy opatřeny povrchovou izolací, 
 které zabraňují vzniku zkratu mezi jednotlivými vinutými cívkami.
 Tato izolace je obyvkle zajištená pomocí speciálních izolačních laků a jejich složení je závislé na konkrétní aplikaci motoru.
 Základní složkou jsou organické pryskyřice, které zajišťují izolační vlastnosti a odolnost proti vysokým teplotám. Další přidané složky
 mohou zvyšovat mechanickou pružnost, odolnost proti chemickým látkám i ochraně proti UV záření.

Tyto cívky jsou pak nejčastěji navinuté na stator motoru do tzv. koncentrovaného uspořádání.
Toto uspořádání nabízí snazší výrobní proces motoru a zároveň 

Konfigurace fází BLDC motoru není standardizována a existuje několik variant. V případě třífázového motoru lze cívky 
zapojit do trojúhelníkového nebo hvězdného uspořádání.
Pojdmě si nyní tyto způsoby představit detailněji.

\secc počet cívek

BLDC motory mohou mít 1-fázové, 2-fázové, 3-fázové a multi-fázové vinutí.
S příbajícím počtem fází se zlepšují jeho vlastnosti, ale také jeho cena a složitost řízení otáček. 
Proto je důležité zvolit správný počet fází, který bude vyhovovat požadavkům konkrétní aplikace.

Jednofázové motory jsou nejjednodušší a nabízí jednoduché řízení otáček. 
Mají však nízký točivý moment a bez použití přidaného kondenzátoru nejsou 
schopny samorozběhu. Další nevýhodou je veliké kolísání točivého momentu 
a možnost rotace pouze v jednom směru. Tyto vlastnosti jsou ale 
dostačující pro aplikace jako pohony pro ventilátory a čerpadla.

Dvoufázové motory nabízí vyšší točivý moment, možnost rotace v obou směrech i schopnost
samorozběhu. Stále však trpí značným kolísáním točivého momentu. Řízení dvoufázových motorů 
je složitější než u jednofázových motorů, ale stále je možné použít jednoduché řídicí metody.
Aplikace, kde se dvoufázové motory nejčastěji využívají jsou například kuchyňské spotřebiče, 
nářadí, ventilátory a čerpadla. 

Třífázové motory jsou nejběžnější volbou pro BLDC motory. 
Oproti předchozím typům motorů je řízení třífázových motorů složitější a vyžaduje
použití speciálních řídicích metod. Tyto metody však umožňují efektivní řízení otáček a točivého momentu motoru.
Třífázové motory se hojně využívají v průmyslové automatizaci, letectví, robotice a v elektromobilních aplikacích. 

Multi-fázové motory jsou nejvýkonnější volbou a oproti předchozím typům nabízí multi-fázové motory vyšší spolehlost. 
Právě díky více fázím je schopen pracovat i při výpadku jedné nebo více fází. 
Tento typ motoru se využívá v aplikacích, kde je vyžadována vysoká spolehlivost a výkon, jako
je například v letectví, kosmonautice a vojenském průmyslu.  

% tabulka pro jaké aplikace se hodí?
\secc vinutí trojúhelník
- méně časté (circulating current -> Při pohybu rotoru BLDC motoru se mění magnetický tok v cívkách a indukuje se napětí. Pokud není tento jev správně kontrolován, může indukované napětí způsobit proudění proudu mezi fázemi motoru. Tento oběhový proud může způsobit ztráty výkonu a zahřívání motoru, což může vést k poklesu účinnosti a životnosti motoru. Aby byl oběhový proud minimalizován nebo eliminován, je důležité provést vhodný návrh motoru a řídicího systému. To zahrnuje optimalizaci geometrie motoru, volbu správného typu cívek a magnetického obvodu, a implementaci pokročilých řídicích algoritmů, které minimalizují indukované napětí a řídí proudy motoru tak, aby byly co nejefektivnější. Takové opatření pomáhají snižovat ztráty výkonu a zvyšovat účinnost BLDC motorů.)
- vyšší počáteční proud (vyšší počáteční točivý moment)
- obtížnější detekce polohy rotoru 

Vinutí do trojúhelníku se u BLDC motorů využívá jen zřídka. Důvodem jsou jeho vlastnosti, které nejsou pro BLDC motory vhodné.
Zapojení do trojúhelníku má vždy uzavřený obvod a vyskytuje se tak tak oběhový proud.
Oběhový proud v BLDC motoru vzniká při změně magnetického toku v cívkách v důsledku pohybu rotoru, což indukuje napětí a způsobuje tak
proud mezi fázemi motoru. Tento jev může způsobit ztráty výkonu a zahřívání motoru, což negativně ovlivňuje jeho účinnost a životnost. 
Pro potlačení tohoto jevu je nutné provést optimalizovaný návrh motoru a řídicího systému. 
Dalším důsledkem uzavřeného obvodu je obtížnější detekce polohy rotoru.
V trojúhelníkové zapojení nastává 30° fázový posun mezi svorkovým a fázovým napětím. Je tedy nezbytné 
při detekci polohy rotoru s touto informací pracovat. Opět tak dochází k vyšším nárokům na návrh motoru.
Další vlastností tohoto zapojení je vyšší počáteční proud při rozběhu motoru. Důsledkem toho motor disponuje 
vyšším počátečním točivým momentem, který může být v některých aplikacích žádaný. 

- uzavřený obvod (oběhový proud)
- vyšší počáteční proud
- obtížnější detekce polohy rotoru


\medskip
\picw=.35\hsize
\clabel[vinuti-trojuhelnik]{Popisek k obrázkům} % to dá popisek do seznamu obrázků
\centerline {\inspic Pictures/3konstrukce/sign-delta.png  \hfil\hfil \inspic Pictures/3konstrukce/coil-delta.png  }\nobreak
%\centerline {a)\hfil\hfil b)}\nobreak\medskip
\caption/f Vinutí cívek do hvězdy ve schématu a při realizaci .
 \medskip

\secc vinutí hvězda
Vinutí do hvězdy představuje průmyslový standard pro BLDC motory, a to z několika důvodů.
Obvod vinutí do hvězdy je otevřený a nevytváří tak oběhový proud. 
Zadruhé, toto uspořádání netrpí posunem fázového a svorkového napětí, což usnadňuje detekci polohy rotoru.
Za druhé, vinutí do hvězdy výrazně snižuje počáteční proudové špičky během startu, 
což zlepšuje spolehlivost a životnost motoru a zároveň má pozitivní vliv na energetickou účinnost.
Tyto vlastnosti z tohoto důvodu činí vinutí do hvězdy preferovanou volbou pro široké spektrum průmyslových
 a komerčních aplikací, kde je kladen důraz na výkon, spolehlivost a účinnost.

- nejčastější vinutí (standart)
- menší počáteční proud (menší počáteční točivý moment)
- snazší detekování polohy rotoru
Civky při zapojení do hvězdy jsou spojeny do jednoho tzv. neutrálního 
bodu. Tento bod je v případě vyvedení na konektoru motoru označen jako N. 


\medskip
\picw=.35\hsize
\clabel[vinuti-hvezda]{Popisek k obrázkům} % to dá popisek do seznamu obrázků
\centerline {\inspic Pictures/3konstrukce/sign-star.png  \hfil\hfil \inspic Pictures/3konstrukce/coil-star-n.png  }\nobreak
%\centerline {a)\hfil\hfil b)}\nobreak\medskip
\caption/f Vinutí cívek do hvězdy ve schématu a při realizaci .
 \medskip


 \sec Permanentní magnety

 \medskip
 \picw=.35\hsize
 \clabel[magnetBLDC+magnetPMSM]{magnet PMSM & BLDC} % to dá popisek do seznamu obrázků
 \centerline {\inspic Pictures/3konstrukce/magnet-BLDC.png \hfil\hfil \inspic Pictures/3konstrukce/magnet-PMSM.png }\nobreak
 \centerline {a) BLDC \hfil\hfil b) PMSM }\nobreak\medskip
 \caption/f Porovnání magnetů BLDC a PMSM motorů.
  \medskip

 - strana 391 BLDC knihy ->

- materiál
- tvar
- umístění

\sec Porovnání zmíněných konstrukcí

 \sec Komutátor

 \sec Řídící elektronika

 - ESC (Electronic Speed Control)

 \sec Chlazení

 