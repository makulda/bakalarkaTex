%\catcode`<=13
%\def<#1>{\hbox{$\langle$\it#1\/$\rangle$}}
%\toksapp\everytt{\catcode`<=13} \toksapp\everyintt{\catcode`<=13}

\chap Teoretický přehled o BLDC Motorech

BLDC\fnote{Anglická zkratka pro brushless direct current motor}, neboli bezkartáčové stejnosměrné motory, jsou typem
synchronního\fnote{Rotor se otáčí synchronně s magnetickým polem statoru.} motoru, který se 
vyznačuje vysokým točivým momentem, vysokou účinností a vynikající regulací otáček. 
To jsou jen některé z vlastností, které dělají BLDC motory vhodnými pro širokou škálu aplikací, od průmyslových pohonů, 
přes elektromobilitu, až po domácí spotřebiče. 

Svými vlastnostmi, možnostmi řízení i konstrukcí se BLDC motory velmi podobají klasickým synchronním motorům s permanentními magnety (PMSM).
Než se tedy pustíme do detailního popisu konstrukce a metod řízení BLDC motorů, bylo by vhodné si vysvětlit rozdíl mezi těmito dvěma typy motorů.
Dále si zdůvodníme, přoč se BLDC motory řadí mezi stejnosměrné motory, ačkoli fungují na střídavém principu jako PMSM. Pro tyto účely musíme
nejprve zmínit historický kontext vývoje BLDC motorů, který jejich označení vysvětlí.

%Ačkoli se v názvu vyskytuje zkratka pro stejnosměrný motor (DC), nelze o něm uvažovat v takovém významu, jako u klasických stejnosměrných motorů.
%V případě použití elektronického komatátoru pro spínání jednotlivých fází se motor chováním velmi podobá klasickým stejnosměrným motorům.
%Elektronický komutátor lze však použít i pro modulaci stejnosměrného napětí na střídavé a motor je schopen pracovat i v takovém řešení. BLDC motory tak lze 
%považovat za hybridní motor, který se nachází někde mezi stejnosměrnými a synchronními motory. 
%Přítomnost DC označení je způsoben historickým kontextem jejich vzniku, podobností v konstrukci a způsobu řízení s klasickými stejnosměrnými motory. 

\sec Historický kontext vývoje BLDC motorů %Vznik BLDC motorů

Stejnosměrné motory s kartáči hrály významnou roli v elektrifikaci během druhé průmyslové revoluce v druhé polovině 19. století. 
Oproti střídavým motorům, včetně PMSM, měli možnost snadné regulace otáček pomocí potenciometru a díky tomu se staly populární volbou pro
průmyslové aplikace, kde byla vyžadována variabilita a přesnost rychlosti. 

Nicméně, tato technologie má své nevýhody. Mechanická změna komutace prostřednictvím kartáčů snižuje účinnost a omezuje životnost
 motoru na 1000-3000 hodin provozu. Odstraněním kartáčů by tak bylo možné vytvořit motor s výrazně delší životností a vyšší účinností. 
 Avšak technologicky nebylo možné takový motor vytvořit. Významný zlom nastal až v 60. letech 20. století s objevem křemíkových polovodičových součástek.
 Tyto součástky umožnily vývoj elektronických komutátorů, které představovaly klíčový krok směrem k nové éře.
 
 V roce 1962 představili T.G. Wilson a P.H. Trickey v článku \textit{D-C machine with solid-state commutation} první funkční řešení bezkartáčového 
 stejnosměrného motoru (BLDC). Název tak vznikl pouze na základě vývojové evoluce DC motorů. \cite [druha] 

\sec Charakteristika BLDC motorů

Ačkoli se řadí do kategorie stejnosměrných motorů, fungují na principu motoru střídavých. A právě z tohoto důvodu se BLDC motory 
mohou vyskytovat v některých pramenech pod označením AC a nikoliv DC (https://www.ti.com/lit/an/sprabz4/sprabz4.pdf).

Konstrukčně se BLDC motory podobají synchronním motorům typu PMSM, ale odlišují se tvarem zpětného indukovaného napětí\fnote{Označovaný též jako {\em back-emf}}.
Zatímco BLDC motory mají trapezoidní tvar zpětného indukovaného napětí, PMSM motory mají sinusoidní. Tento tvar zpětného indukovaného napětí
je důsledkem charakteristiky použitých permanentních magnetů a způsobu vinutí cívek v motoru.


Charakteristickým znakem BLDC motorů je trapezoidní tvar zpětného indukovaného napětí\fnote{Označovaný též jako {\em back-emf}}, kterým 
se liší od konstrukčně téměř identických PMSM klasických synchronních motorů se sinusoidním tvarem zpětného indukovaného napětí. 
Trapezoidní tvar zpětného indukovaného napětí
je důsledkem charakteristiky použitých permanentních magnetů a způsobu vinutí cívek v motoru.

BLDC motory jsou konstrukčně téměř identické s moty typu PMSM\fnote{permanent magnet synchronous motor}.

\sec Rozdíl mězi BLDC a PMSM motory

Hlavní výhodou BLDC motorů oproti ostatním typům motorů je jednoduchá konstrukce, která 
BLDC motory nabizí hned několik konstrukčních variant, které se liší v počtu fází i způsobu řízení. 
Nejběžnějšími konstrukčními variantami jsou 1-fázové, 2-fázové, 3-fázové a multi-fázové motory.
Tím se liší od klasických synchronních motorů, které mají většinou 3 fáze.
- trapezoidní back-emf
- BLDC podobné PMSM
- proud teče jen jednou fází
- vysoký točivý moment
- snadné řízení
- účinnost až 90\%

