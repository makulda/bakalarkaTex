%\catcode`<=13
%\def<#1>{\hbox{$\langle$\it#1\/$\rangle$}}
%\toksapp\everytt{\catcode`<=13} \toksapp\everyintt{\catcode`<=13}

\chap Teoretický přehled o BLDC motorech

% úvod do kapitoly
Tato kapitola se zaměčuje na popis BLDC motorů z pohledu charakteristiky a vlastností. Bude zde zmíněna i historie jejich vzniku
a vysvětlen rozdíl mezi motorem typu BLDC a PMSM.

% charakteristika BLDC motorů (podobnost s DC motory, rozdíly)
\sec Charakteristika
% že jsou bezkartáčové, rozdíly oproti DC motorům, kam se řadí TODO: koncentrované vinutí
BLDC\fnote{Anglická zkratka pro brushless direct current motor}, neboli bezkartáčový stejnosměrný motor, je moderní
typ elektrického motoru, který je svými vlastnostmi a konstrukcí podobný klasickým stejnosměrným motorům s kartáči, též označovanými jako DC.
Oproti DC motorům se však liší absencí mechanického komutátoru\fnote{Součástka sloužící k přepínání polarit proudů v jednotlivých cívkách motoru. 
Obsahuje tzv. kartáče, které se třou o lamely. Tímto kontaktem prochází proud do motoru.}, který je nahrazen komutátorem elektronickým. Odtud pochází 
jeho název - bezkartáčový. 

BLDC motory se řadí do kategorie synchronních motorů. Označení synchronní znamená, že pohyb rotoru motoru je synchronní s
pohybem magnetického pole statoru a nedochází tak k tzv. kluzným otáčkám, které jsou typické pro asynchronní motory. 

% tabulka DC, AC?

\secc Vznik BLDC motorů
% že vznikly jako náhrada za klasické DC motory
Vznik BLDC motorů je úzce spojen s vývojem polovodičových technologií v 60. letech 20. století, které 
umožnili výrobu spolehlivých a výkonných elektronických komutátorů. V roce 1962 T.G. Wilson a P.H. Trickey publikovali
ve své práci \textit{D-C machine with solid-state commutation} první návrh a realizaci BLDC motoru s elektronickým komutátorem
a jsou tak označováni za jeho vynálezce.

\secc Vlastnosti
% vysoká účinnost, vysoké otáčky, vysoký výkon, vysoká hustota výkonu, vysoká regulace otáček, vysoká životnost, vysoká cena, složitější řízení, nejiskří -> možno použít v nebezpečném prostředí
Mezi charakteristické vlastnosti BLDC motorů patří vysoká účinnost, vysoký točivý moment, dlouhá životnost a výborná regulace otáček.

BLDC motory dosahují účinnosti až 90 \% a při vhodné konstrukci dosahují vysokých otáček blížících se až k 100 000 ot/min. % Permanent magnet motor technology str. 441
Vyznačují se také jednoduchou a lehkou konstrukcí, díky které disponují vysokou hustotou výkonu\fnote{Výkon v poměru s hmotností a objemem motoru.}. 

Dále se vyznačují variabilitou konstrukčních řešení a lze tak nelézt BLDC motory v různých provedeních a rozměrech.

\secc Back-emf

Při pohybu rotoru s permanentními magnety v blízkosti vodivých cívek statoru dochází k indukci napětí na cívkách.
Toto napětí bývá označováno jako zpětné indukované napětí, nebo zkráceně back-emf\fnote{Zkratka anglického výrazu \textit{Back elektromotive force}}.
Tento jev je přítomný u všech synchronních motorů. Typickým znakem BLDC motorů je trapezoidní tvar zpětného indukovaného 
napětí. % Tento tvar je způsoben 

\medskip
 \picw=14 cm \cinspic Pictures/2teorie/bemf-graph.png 
 \caption/f Trapezoidní tvar zpětného indukovaného napětí
 \medskip

Tento tvar je způsoben použitím koncentrovaného vinutí statorových cívek a permanentních magnetů s vnitřním magnetickým tokem 
kolmým k povrchu rotoru \fnote{Pokud jsou použity magnety s paralelní magnetizací, vzniká sinusoidální zpětné elektromotorické napětí a nejedná se o motor BLDC, ale motor typu BLAC}.

\medskip
 \picw=7 cm \cinspic Pictures/2teorie/magnet-BLDC.png 
 \caption/f Magnetický tok uvnitř permanentních magnetů BLDC motoru
 \medskip


%https://electronics.stackexchange.com/questions/428028/why-is-back-emf-of-bldc-trapezoidal

\sec Rozdíl mezi BLDC a PMSM motory

PMSM \fnote{Zkratka anglického výrazu \textit{Permanent Magnet Synchronous Motor}} motory jsou dalším typem synchronních motorů. 
S BLDC motory sdílejí mnoho společných vlastností a konstrukčních podobností. Z tohoto důvodu
je často obtížné odlišit tyto dva typy motorů. Rozlišit je lze dle typu vinutí statorových cívek.
PMSM motory využívají vinutí distribuované, zatímco BLDC motory využívají vinutí koncentrované.
To má za následek rozdílný tvar zpětného indukovaného napětí, které je u PMSM motorů sinusoidní.
% Comparing the Performance of Parallel Multi-Phase Brushless DC Motors: A Comprehensive Analysis
% electric motors and drives str. 357

\sec Zařazení mezi AC nebo DC motory?

Zařazení BLDC motorů do kategorie střídavých nebo stejnosměrných motorů není jednoznačné, neboť 
mohou být napájeny jak stejnosměrným tak i střídavým napětím. % electric motors and drives str. 357

Některé prameny zařazují BLDC motory mezi stejnosměrné z důvodu použití stejnosměrného zdroje napětí nebo konstrukční podobností 
s DC motory. 

Argumentem zařazení BLDC motorů mezi AC motory je velmi podobný princip fungování a charakteristika se synchronními AC motory.

Z tohoto důvodu nelze BLDC motory do této kategorie jednoznačně zařadit, a proto se v pramenech často objevují pod různými označeními.
%(https://www.ti.com/lit/an/sprabz4/sprabz4.pdf).

