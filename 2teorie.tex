%\catcode`<=13
%\def<#1>{\hbox{$\langle$\it#1\/$\rangle$}}
%\toksapp\everytt{\catcode`<=13} \toksapp\everyintt{\catcode`<=13}

\chap Teoretický přehled o BLDC Motorech

BLDC, neboli Brushless DC motor, je typ synchronního motoru, který se vyznačuje vysokým točivým momentem, vysokou účinností a vynikající regulací otáček.
Ačkoli se v názvu vyskytuje zkratka pro stejnosměrný motor (DC), jedná se v podstatě o motor střídavý. Tento fakt je důsledkem
konstrukce, která se podobá konstrukci synchronních motorů a okolnostem vzniku jejich vývoje. 
Než se pustíme do detailního popisu konstrukce a metod řízení BLDC motorů, bylo by vhodné si nejprve tyto okolnosti vzniku přiblížit a 
a zařadit BLDC motory do širšího kontextu. 
%které vedly k vzniku BLDC motorů a posléze i o charakteristických vlastnostech, kterými se odlišují od jiných typů motorů.

\sec Okolnosti vzniku BLDC motorů %Vznik BLDC motorů

    Stejnosměrné motory s kartáči hrály klíčovou roli v elektrifikaci během druhé průmyslové revoluce v druhé polovině 19. století. 
Díky jejich schopnosti snadné regulaci otáček pomocí potenciometru se staly populární volbou pro různé průmyslové aplikace, kde byla
 vyžadována variabilita a přesnost rychlosti.

Nicméně, tato technologie má své nevýhody. Mechanická změna komutace prostřednictvím kartáčů snižuje účinnost a omezuje životnost
 motoru na 1000-3000 hodin provozu. Odstraněním kartáčů by tak bylo možné vytvořit motor s výrazně delší životností a vyšší účinností. 
 Avšak technologicky nebylo možné takový motor vytvořit. Významný zlom nastala až v 60. letech 20. století s objevem křemíkových polovodičových součástek.
 Tyto součástky umožnily vývoj elektronických komutátorů, které představovaly klíčový krok směrem k nové éře.
 
 V roce 1962 představili T.G. Wilson a P.H. Trickey v článku \textit{D-C machine with solid-state commutation} první funkční řešení bezkartáčového 
 stejnosměrného motoru (BLDC) a otevřeli tak cestu pro vývoj moderních BLDC motorů. \cite [druha].

 \sec Princip fungování BLDC motorů
 - proč je DC když je to AC

\sec Charakteristika BLDC motorů

BLDC motory se řadí do kategorie synchronních\fnote{Rotor se otáčí synchronně s magnetickým polem statoru.} motorů.
Díky své vysoké hustotě výkonu, účinnosti a vynikající regulaci otáček se staly populární volbou pro širokou škálu aplikací.

Charakteristickým znakem BLDC motorů je trapezoidní tvar zpětného indukovaného napětí\fnote{Označovaný též jako {\em back-emf}}, kterým 
se liší od klasických synchronních motorů se sinusoidním tvarem zpětného indukovaného napětí. 

BLDC motory nabizí hned několik konstrukčních variant, které se liší v počtu fází i způsobu řízení. 
Nejběžnějšími konstrukčními variantami jsou 1-fázové, 2-fázové, 3-fázové a multi-fázové motory.
Tím se liší od klasických synchronních motorů, které mají většinou 3 fáze.



Díky napájení DC napětím jsou otáčky snadno řiditelné 
Tyto vlastnosti z nich činí  


- proud teče jen jednou fází
- vysoký točivý moment
- snadné řízení
- účinnost 




- zařazení (synchronní, DC,) 
- vysoký točivý moment
- trapezoidní back-emf
BLDC motory se vyznačují několika klíčovými vlastnostmi, které je činí vhodnými pro širokou škálu aplikací. Mezi ty největší patří 
- zařazení do kategorie synchronních motorů (obrázek, kam se řadí, takový to schéma)

