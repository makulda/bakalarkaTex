\catcode`<=13
\def<#1>{\hbox{$\langle$\it#1\/$\rangle$}}
\toksapp\everytt{\catcode`<=13} \toksapp\everyintt{\catcode`<=13}

\chap Teoretický přehled o BLDC Motorech

\sec Vznik BLDC motorů

    Stejnosměrné motory s kartáči hrály klíčovou roli v elektrifikaci během druhé průmyslové revoluce v druhé polovině 19. století. 
Díky jejich schopnosti snadné regulace otáček pomocí potenciometru se staly populární volbou pro různé průmyslové aplikace, kde byla
 vyžadována variabilita a přesnost rychlosti.

Nicméně, tato technologie má své nevýhody. Mechanická změna komutace prostřednictvím kartáčů snižuje účinnost a omezuje životnost
 motoru na 2000-3000 hodin provozu. Odstraněním kartáčů by tak bylo možné vytvořit motor s výrazně delší životností a vyšší účinností. 
 Avšak technologicky nebylo možné takový motor vytvořit. Významný zlom nastala až v 60. letech 20. století s objevem křemíkových polovodičových součástek.
 Tyto součástky umožnily vývoj elektronických komutátorů, které představovaly klíčový krok směrem k nové éře.
 
 V roce 1962 představili T.G. Wilson a P.H. Trickey v článku \textit{D-C machine with solid-state commutation} první funkční řešení bezkartáčového 
 stejnosměrného motoru (BLDC) a otevřeli tak cestu pro vývoj moderních BLDC motorů. \cite[prvni].

 \sec charakteristika BLDC motorů
    - zařazení (synchronní, DC,) 
    - vysoký točivý moment
    BLDC motory se vyznačují několika klíčovými vlastnostmi, které je činí vhodnými pro širokou škálu aplikací. Mezi ty největší patří 
    - zařazení do kategorie synchronních motorů (obrázek, kam se řadí, takový to schéma)
