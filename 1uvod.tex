% Lokální makra patří do hlavního souboru, ne sem.
% Tady je mám výjimečně proto, že chci nechat hlavní soubor bez maker,
% která jsou jen pro tento dokument. Uživatelé si pravděpodobně budou
% hlavní soubor kopírovat do svého dokumentu.

\chap Úvod

% základní informace ~

% elektrické pohony spotřebují více jak 40 % vyrobené elektřiny (https://ieeexplore-ieee-org.ezproxy.techlib.cz/stamp/stamp.jsp?tp=&arnumber=6377525) (https://www.researchgate.net/figure/Projected-global-electric-motor-system-electricity-consumption_fig1_254439185)
% hrají klíčovou roli v rychle se rozvíjejícím světě

V současném průmyslovém a technologickém prostředí hrají elektromotory klíčovou roli. Poskytují pohon pro širokou 
škálu zařízení od drobných domácích spotřebičů až po komplexní průmyslové systémy.
Na provoz těchto zařízení se spotřebuje více jak 40 \% vyrobené elektrické energie na celém světě,
což je pro porovnání dvakrát více než spotřeba pro veškeré osvětlení. 
Elekromotory se tak staly nedílnou součástí moderního světa a jejich význam stále roste.
S tímto trendem souvisí i rostoucí důraz na energetickou efektivitu a udržitelnost, což vede k hledání nových technologií a
inovací v oblasti elektrických pohonů. Přesně tyto vlastnosti nabízejí bezkartáčové stejnosměrné motory (BLDC) a stávají se tak
preferovanou volbou v mnoha odvětvích. Mimo jiné nabízí i vysokou hustotu výkonu, výbornou 
regulaci otáček a účinnost patřící k nejvyšším mezi elektrickými motory.

S pokračujícím rozvojem technologií řízení a senzoriky bude možné ještě zvýšit jejich spolehlivost a přesnost v rámci 
složitých automatizovaných systémů. Tyto trendy naznačují, že BLDC motory mají před sebou 
slibnou budoucnost a budou hrát stále významnější roli v moderním průmyslu a technologickém prostředí.

% cíl práce
Hlavním záměrem bakalářské práce je detailně analyzovat konstrukci bezkartáčových stejnosměrných motorů a porozumět 
jim z fyzikálního hlediska. Následně bude podrobně rozebrána škála možných řídících algoritmů a jejich charakteristické
vlastnosti. Dalším cílem této práce bude vytvořit model BLDC motoru v prostředí \nobreak{MATLAB/Simulink} s implementací \nobreak{six-stepového}
řídícího algoritmu v proudovém režimu. Tento model bude vytvořen s využitím fyzikálních principů a parametrů motoru od 
firmy LINIX.

V neposlední řadě vytvořený model bude sloužit jako referenční vzor pro vytvoření implementace six-stepového algoritmu pro evaluční 
desku STEVAL-\nobreak{}SPIN3202 od firmy STMicroelectronics a motor 45ZWN24-40 od firmy LINIX. V rámci práce bude vytvořeno i uživatelské rozhraní, které 
umožní snadné ovládání motoru a nastavení parametrů regulátoru. Vznikne tak 
praktický nástroj pro využití v konkrétních aplikacích. 

%struktura práce
Tato práce přináší přínos prostřednictvím hloubkového studia struktury a chování BLDC motoru, a to s důrazem na jejich praktické 
využití v různých aplikacích. Vytvoření modelu a knihovny pro evaluční desku může sloužit jako praktický nástroj pro vývojáře a studenty, kteří
by chtěli navázat na tuto práci, či využít vytvořenou knihovnu pro konkrétní aplikaci.




