% Lokální makra patří do hlavního souboru, ne sem.
% Tady je mám výjimečně proto, že chci nechat hlavní soubor bez maker,
% která jsou jen pro tento dokument. Uživatelé si pravděpodobně budou
% hlavní soubor kopírovat do svého dokumentu.

\chap Úvod

% základní informace ~

% elektrické pohony spotřebují více jak 40 % vyrobené elektřiny (https://ieeexplore-ieee-org.ezproxy.techlib.cz/stamp/stamp.jsp?tp=&arnumber=6377525) (https://www.researchgate.net/figure/Projected-global-electric-motor-system-electricity-consumption_fig1_254439185)
% hrají klíčovou roli v rychle se rozvíjejícím světě

V současném průmyslovém a technologickém prostředí hrají elektromotory klíčovou roli. Poskytují pohon pro širokou 
škálu zařízení od domácích spotřebičů až po průmyslovou automatizaci.
Na provoz těchto zařízení se spotřebuje více jak 40 \% vyrobené elektrické energie na celém světě,
což je pro porovnání dvakrát více než spotřeba pro veškeré osvětlení. 
Elekromotory se tak staly nedílnou součástí moderního světa a jejich význam stále roste.
Stejně tak roste zastoupení synchronních bezkartáčových stejnosměrných motorů (BLDC) mezi elektrickými motory.
Ty se stávají preferovanou volbou v mnoha odvětvích díky svým vlastnostem jako je vysoká hustota výkonu, výborná 
regulace otáček a účinnosti patříci k nejvyšším mezi elektrickými motory.
BLDC motory také nabízejí výhody v podobě nižší hlučnosti a delší životnosti ve srovnání s tradičními kartáčovými 
stejnosměrnými motory. Jejich použití se rozšiřuje do nových oblastí, jako jsou elektrická vozidla a autonomní systémy. 
Tento trend podtrhuje rostoucí důležitost výzkumu a vývoje v oblasti BLDC technologií a jejich integrace do moderních systémů pohonu.

BLDC motory také přinášejí významné úspory energie, což je v souladu s trendem k udržitelnějšímu 
provozu a ochraně životního prostředí. Jejich využití v průmyslových aplikacích přispívá k efektivnější výrobě a snižuje náklady na provoz. 
Navíc, s rozvojem technologií řízení a senzoriky se zvyšuje jejich spolehlivost a přesnost ve složitých systémech automatizace. 




Slouží jako pohon pro širokou škálu zařízení od jednoduchých domácích spotřebičů
až po komplexní průmyslová zařízení.
Ve všech těchto odvětvích se využívají i synchronní bezkartáčové stejnosměrné motory (BLDC) a to díky
svým vlastnostem jako je vysoká hustotu výkonu, výborná regulace otáček a účinnosti patříci k nejvyšším 
mezi elektrickými motory. 


  Tyto motory nabízejí jedinečné vlastnosti, které je činí preferovanou volbou v mnoha odvětvích. 
  Patří mezi ně vysoká účinnost, což je zvláště důležité v období rostoucího důrazu na energetickou 
  efektivitu a udržitelnost. Dále se vyznačují dlouhou životností, nízkou hlučností a schopností 
  přesné řiditelnosti, což je klíčové zejména v průmyslových aplikacích, kde je nezbytné dosáhnout 
  přesných otáček a dosahovat vysokých otáček. Tyto vlastnosti činí BLDC motory klíčovými aktéry 
  v elektrických pohonech, a to nejen v průmyslu, ale i v oblasti elektromobility, spotřebičů, robotiky
   a dalších odvětvích, kde je kladen důraz na výkonnost, spolehlivost a účinnost.

A ve všech těchto odvětvích se BLDC motory využívají a 
jejich význam stále roste.


V současném průmyslovém a technologickém prostředí hrají elektromotory klíčovou roli, poskytující pohon pro širokou 
škálu zařízení a aplikací od domácích spotřebičů po průmyslová zařízení. V kontextu elektrických pohonů se významným krokem 
vpřed stává využití synchroních bezkartáčových stejnosměrných motorů (BLDC). Jejich význam spočívá 
v jedinečných vlastnostech, které se staly důvodem, proč jsou preferovanou volbou v mnoha odvětvích. 
Mezi tyto vlastnosti patří vysoká účinnost, což představuje efektivní využití dodávané elektrické energie, a to je zvláště 
důležité v období rostoucího důrazu na energetickou efektivitu a udržitelnost. Dále se vyznačují dlouhou životností, díky které se
 tak snižují náklady na údržbu, a nízkou hlučností, která je klíčová v aplikacích, kde je potřeba minimalizovat akustický dopad. 
Kromě toho je jejich schopnost přesné řiditelnosti důležitým faktorem, zejména v průmyslových aplikacích, kde je nezbytné 
dosáhnout přesných otáček a také excelují v možnosti dosahovat vysokých otáček. Tyto vlastnosti činí BLDC motory klíčovými aktéry v elektrických pohonech, a to 
nejen v průmyslu, ale i v oblasti elektromobility, spotřebičů, robotiky a dalších odvětvích, kde je kladen důraz na 
výkonnost, spolehlivost a účinnost.

% cíl práce
Hlavním záměrem bakalářské práce je detailně analyzovat konstrukci bezkartáčových stejnosměrných motorů a porozumět 
jim z fyzikálního hlediska. Následně bude podrobně rozebrána škála možných řídících algoritmů a jejich charakteristické
vlastnosti. Dalším cílem této práce bude vytvořit model BLDC motoru v prostředí~\nobreak{}MATLAB/Simulink s implementací \nobreak{}six-stepového
řídícího algoritmu v proudové režimu. Tento model bude vytvořen s využitím fyzikálních principů a parametrů motoru od 
firmy LINIX.

V neposlední řadě, vytvořený model bude sloužit jako referenční vzor pro vytvoření implementace six-stepového algoritmu pro evaluční 
desku STEVAL-\nobreak{}SPIN3202 od firmy STMicroelectronics a motor 45ZWN24-40 od firmy LINIX. V rámci práce bude vytvořeno i uživatelské rozhraní, které 
umožní snadné ovládání motoru a nastavení parametrů regulátoru. Vzikne tak 
praktický nástroj pro využití v konkrétních aplikacich. 

%struktura práce
Tato práce přináší přínos prostřednictvím hloubkového studia struktury a chování BLDC motorů, a to s důrazem na jejich praktické 
využití v různých aplikacích. Vytvoření modelu a knihovny pro evaluční desku může sloužit jako praktický nástroj pro vývojáře a studenty, kteří
by chtěli navázat na tuto práci, či využít vytvořenou knihovnu pro konkrétní aplikaci. Tímto způsobem se práce snaží přispět k rozvoji a optimalizaci elektrických 
pohonů, které mají stále rostoucí význam v elektrifikovaném světě.




