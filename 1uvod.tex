% Lokální makra patří do hlavního souboru, ne sem.
% Tady je mám výjimečně proto, že chci nechat hlavní soubor bez maker,
% která jsou jen pro tento dokument. Uživatelé si pravděpodobně budou
% hlavní soubor kopírovat do svého dokumentu.

\def\ctustyle{{\ssr CTUstyle}}
\def\ttb{\tt\char`\\} % pro tisk kontrolních sekvencí v tabulkách

\chap Úvod

% základní informace ~

V současném průmyslovém a technologickém prostředí hrají elektromotory klíčovou roli, poskytujíce pohon pro širokou 
škálu zařízení a aplikací, od domácích spotřebičů po průmyslová zařízení. V tomto rámci představuje jeden z výrazných 
vývojových kroků v oblasti elektrických pohonů zavedení bezkartáčových stejnosměrných motorů (BLDC). Jejich význam spočívá 
v jedinečných vlastnostech, které se staly důvodem, proč jsou preferovanou volbou v mnoha odvětvích. Mezi tyto vlastnosti 
patří vysoká účinnost, což znamená efektivnější využití dodávané elektrické energie, což je zvláště důležité v době 
rostoucího důrazu na energetickou účinnost a udržitelnost. Dále se vyznačují dlouhou životností, což snižuje náklady 
na údržbu a výměnu, a nízkou hlučností, což je klíčové v aplikacích, kde je potřeba minimalizovat akustický dopad. 
Kromě toho je jejich schopnost přesné řiditelnosti důležitým faktorem, zejména v průmyslových aplikacích, kde je nezbytné 
dosáhnout přesných a spolehlivých pohybů. Tyto vlastnosti činí BLDC motory klíčovými aktéry v elektrických pohonech, a to 
nejen v průmyslu, ale i v oblasti elektromobilita, spotřebičů, robotiky a dalších odvětvích, kde je kladen důraz na 
výkonnost, spolehlivost a účinnost.

% cíl práce
Hlavním záměrem bakalářské práce je detailně analyzovat konstrukci bezkartáčových stejnosměrných motorů a porozumět 
jim z fyzikálního hlediska. Následně bude podrobně rozebrána škála možných řídících algoritmů a jejich charakteristické
vlastnosti. Dalším cílem této práce bude vytvořit model BLDC motoru v prostředí~\nobreak{}MATLAB/Simulink s implementací \nobreak{}six-stepového
řídícího algoritmu v proudové režimu. Tento model bude vytvořen s využitím fyzikálních principů a parametrů motoru od 
firmy LINIX.

V neposlední řadě, vytvořený model bude sloužit jako referenční vzor pro vytvoření six-stepové knihovny. Tato knihovna bude 
přizpůsobena pro evaluční desku STEVAL-\nobreak{}SPIN3202 od firmy STMicroelectronics a motor 45ZWN24-40 od firmy LINIX. V rámci práce bude vytvořeno i uživatelské rozhraní, které 
umožní snadné ovládání motoru a nastavení parametrů regulátoru. Vzikne tak 
praktický nástroj pro využití v konkrétních aplikacich. 

%struktura práce
Tato práce přináší přínos prostřednictvím hloubkového studia struktury a chování BLDC motorů, a to s důrazem na jejich praktické 
využití v různých aplikacích. Vytvoření modelu a knihovny pro evaluční desku může sloužit jako praktický nástroj pro vývojáře a studenty, kteří
by chtěli navázat na tuto práci, či využít vytvořenou knihovnu pro konkrétní aplikaci. Tímto způsobem se práce snaží přispět k rozvoji a optimalizaci elektrických 
pohonů, které mají stále rostoucí význam v elektrifikovaném světě.




