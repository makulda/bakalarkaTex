
\catcode`<=13
\def<#1>{\hbox{$\langle$\it#1\/$\rangle$}}
\toksapp\everytt{\catcode`<=13} \toksapp\everyintt{\catcode`<=13}

\chap Konstrukce BLDC motoru

Bezkartáčový stejnosměrný motor (BLDC) reprezentuje pokročilý typ 
elektromotoru, jehož konstrukce se skládá z dvou klíčových částí -- 
rotoru a statoru. Stator, často označovaný jako neměnná nebo 
nepohyblivá část, hraje zásadní roli v tom, jak elektromotor funguje. 
Je to pevná struktura, která obsahuje vinuté cívky a generuje tak magnetické pole.

Naopak rotor představuje pohyblivou část motoru, která koná rotační 
pohyb kolem stacionárního statoru. Rotor může bývá osazen permanentními
 magnety, které reagují na magnetické pole 
 generované statorovými cívkami. Tato interakce mezi státorem a 
 rotorem umožňuje přeměnu elektrické energie na mechanický pohyb.

\sec Stator

Stator je největší částí motoru a je tvořen statorovými cívkami,
    které jsou umístěny v železném jádře. Statorové cívky jsou 
    uspořádány do trojúhelníkového nebo hvězdicového zapojení. 
    Většina BLDC motorů má tři statorové cívky, které jsou 
    umístěny v 120\,stupňových intervalech. Tento typ BLDC motoru
    se nazývá trojfázový. 
